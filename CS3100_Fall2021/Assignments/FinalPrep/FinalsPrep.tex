
% --------------------------------------------------------------
% This is all preamble stuff that you don't have to worry about.
% Head down to where it says "Start here"
% --------------------------------------------------------------

\documentclass[12pt]{article}

\usepackage{graphicx,url}
\usepackage{proof}
\usepackage{framed}
\usepackage{etaremune}

\usepackage[margin=1in]{geometry}
\usepackage{amsmath,amsthm,amssymb,amsfonts}
\usepackage{paralist}
\thispagestyle{empty}

% 1. To get version suitable for students to populate,
%    remove the contents of the \ignoreSoln{..body..}
%
% 2. To get a version suitable for generating PDF 
%    without solutions, remove the #1 below
%
% 3. To generate solutions, keep the #1 below
%
% 4. Assigned grader fills \ignoreSoln{..body..}
%    and also provides his/her feedback to student
%    and policy followed for point deduction
%    So design policy before grading begins.

\newcommand{\ignoreSoln}[1]{#1}   
%\newcommand{\ignoreModel}[1]{#1} 


\newcommand{\bigset}[2]{\big\{\;#1\;:\;#2\;\big\}}
\newcommand{\N}{\mathbb{N}}
\newcommand{\Z}{\mathbb{Z}}
\newcommand{\R}{\mathbb{R}}
\newcommand{\Np}{\mathbb{N^{+}}}

\newenvironment{theorem}[2][Theorem]{\begin{trivlist}
\item[\hskip \labelsep {\bfseries #1}\hskip \labelsep {\bfseries #2.}]}{\end{trivlist}}
\newenvironment{lemma}[2][Lemma]{\begin{trivlist}
\item[\hskip \labelsep {\bfseries #1}\hskip \labelsep {\bfseries #2.}]}{\end{trivlist}}
\newenvironment{exercise}[2][Exercise]{\begin{trivlist}
\item[\hskip \labelsep {\bfseries #1}\hskip \labelsep {\bfseries #2.}]}{\end{trivlist}}
\newenvironment{reflection}[2][Reflection]{\begin{trivlist}
\item[\hskip \labelsep {\bfseries #1}\hskip \labelsep {\bfseries #2.}]}{\end{trivlist}}
\newenvironment{proposition}[2][Proposition]{\begin{trivlist}
\item[\hskip \labelsep {\bfseries #1}\hskip \labelsep {\bfseries #2.}]}{\end{trivlist}}
\newenvironment{corollary}[2][Corollary]{\begin{trivlist}
\item[\hskip \labelsep {\bfseries #1}\hskip \labelsep {\bfseries #2.}]}{\end{trivlist}}

\DeclareMathSizes{14}{14}{14}{14}

\begin{document}

% --------------------------------------------------------------
%                         Start here
% --------------------------------------------------------------

%\renewcommand{\qedsymbol}{\filledbox}


\begin{center}
\begin{large}
  CS 3100, Fall 2021, Finals Prep, Dec 8, 2021
  \ \\
\end{large}


\end{center}
\date{}


\begin{enumerate}

\item Show that $Y = (\lambda f. \; (lambda x.\; f(x\; x))\; (lambda x.\; f(x\; x)) )$
   is a fixed-point combinator.

 \item Show that $Ye$
   is a fixed-point combinator for eager languages.
 \item[] $Y_e G = (\lambda f. (\lambda x. (xx) [\lambda y. f(\lambda v. ((yy )v))])G$.

\item Write a recursive function that sums from $N$ down to $1$ and
  express its using the $Ye$ combinator. Demonstrate, by running, the series summation
  from a few numbers (e.g., 10, 50, 100) down to 1.

\item Is \( L_{emptycfg} =
    \{ \langle G\rangle \;:\; L(G)=\emptyset \}\) recursive? Here,
    $G$ is a CFG.

    \begin{compactenum}
    \item[] {\sf Answer: Yes. Keep sweeping from the start symbol of $G$
      to see if $G$ gets caught up in infinite recursion. Actually sweep
      from the terminals back to the $S$ (start) symbol of $G$. Somehow
      if $G$ manages to produce any string $s$ at all, then $L(G)\neq\emptyset$.
      Else it is equal to $\emptyset$.
      }
    \end{compactenum}


 \item Is \( L_{G1eqG2} =
    \{ \langle G_1,G_2\rangle \;:\; L(G_1)= L(G_2) \}\) RE, where
    $G_1$ and $G_2$
    are CFGs?

    \begin{compactenum}
    \item[] {\sf Answer: No.
      If you find a string $w$ accepted by one grammar but not the other, then
      the grammars are not language-equivalent.
      But if you keep finding all $w$ within the language of both grammars, you can't stop.
      Intuitively, grammar equality is not established with one string accepted by both.
      One has to ``exhaust'' all strings.
      The formal proof was discussed as a canvas response but not needed.
      }
    \end{compactenum}


  \item Prove that there exist non-RE languages.
    \begin{sf}
      Argue how each RE language can be represented by a bit-vector, calling
      the bits

      \begin{footnotesize}
\begin{verbatim}
L0  b00 b01 b01 ...
L1  b10 b11 b12 ...
L2  b21 b22 ...
L3 ...
\end{verbatim}
      \end{footnotesize}
      Now consider the complemented diagonal. Write it out. Argue
      how it compares with each of the languages L0, L1, ...
    \end{sf}
    \begin{compactenum}
    \item[] {\sf Detailed Answer: Attempt to show that all TMs can be
      ``numbered'' (listed out). Thus, all the RE languages can be numbered.
      That is what I mean by L0, L1, L2, ... above.
      But now, one can find ONE language not in ANY of these languages.
      This is discussed in Appendix-C of the book (last ``chapter'').}
    \end{compactenum}


    %=================================================================
    
  \item Prove that the set of Turing  machines $T$ that halt when
    started with input string $w$, and
    that do not write outside of positions $p_1$
    and $p_2$ on the tape ($p_1\leq p_2$) is recursive.
  \item[]
    \begin{sf}
      Answer:
    \begin{compactenum}
    \item Notice that the TM is not allowed to write more than
      a finite number of cells.
    \item Keep simulating this TM, observing the IDs attained
      by the TM.
    \item The number of IDs is finite. So when the IDs begin repeating
      and the TM has not yet halted, we can conclude that the
      TM will never halt.
    \item Following the aforesaid algorithm, we can conclude
      that this set of $\langle T,w,p_1,p_2\rangle$ is recursive.
    \end{compactenum}
    \end{sf}

        %=================================================================
    
  \item Argue that
    RE sets are closed under union, concatenation, intersection, reversal.
  Here is a taste for one proof (closure under intersection); write the others:

  \begin{sf}
    Answer:
  \begin{compactitem}
  \item Define the following TM that takes $M_1$ and $M_2$:
      \begin{compactitem}
  \item intersectM1M2($M_1$,$M_2$):
  \item run $M_1$ and $M_2$ in dovetail order on inputs, also
    generated by the dovetail enumeration procedure
  \item When this simulation notices $M_1$ and $M_2$ accepting
    some string, it emits that string
  \item This procedure now lists all strings in $M_1$ and $M_2$.
      \end{compactitem}
  \end{compactitem}
  \end{sf}

  %=================================================================

\item Present and prove a mapping reduction from
  $A_{TM}$ to $CFL_{TM}$.

    %=================================================================
    
  \item 
  Do Problem 5, Page 232, Exercise 15.2.3 of our book.
  This asks you to perform
  a mapping reduction from the PCP to the CFG Ambiguity problem,
  and thus argue that if there is a decider for the CFG Ambiguity problem,
  then there will be a decider for PCP also.
  %
  The languages involved in this exercise are now formally defined:
  %
  PCP is defined on Page 229, Section 15.2, and reproduced below:\\
Given any alphabet $\Sigma$ such that $|\Sigma| > 1$, consider the
{\em tile alphabet} ${\cal T}\subseteq \Sigma^{+} \times \Sigma^{+}$.
%
Define
\[ PCP = \{ S \;:\; S \; {\rm is\; a\; finite\; sequence\;
\;of
\;elements
\;over
\; {\cal T}\; that\; has\; a\; solution}\}.\]

Also, define
    \[ Amb = \{ \langle G\rangle \;:\; G \;{\rm is}\; {\rm a}\; CFG
    \;{\rm that}\; \;{\rm has}\; > 1 \;{\rm parse}\; \;{\rm for}\;
    \;{\rm some}\; w\in\Sigma^{*}\}\]


  %---
  \begin{enumerate}
\item[] \label{pcpambmr}
    Here is how we can build a mapping reduction from
PCP to Amb; please fill in missing steps
(if any) and argue that the mapping reduction actually works
(achieves its purpose).
\item[]
Let
\[ A = w_1,w_2,\ldots,w_n \]
and
\[ B = x_1,x_2,\ldots,x_n \]
be two lists of words over a finite alphabet $\Sigma$.
Let $a_1,a_2,\ldots,a_n$ be symbols that do not appear in any of 
the $w_i$ or $x_i$.
Let $G$ be a CFG
\[ (\{S,S_A,S_B\},\; \Sigma\cup\{a_1,\ldots,a_n\},\; P, \; S), \]
where $P$ contains the productions
\begin{itemize}
  \item[] $S\rightarrow S_A$,
  \item[] $S\rightarrow S_B$,
  \item[] For $1\leq i\leq n$, $S_A\rightarrow w_i S_A a_i$,
  \item[] For $1\leq i\leq n$, $S_A\rightarrow w_i a_i$,
  \item[] For $1\leq i\leq n$, $S_B\rightarrow x_i S_B a_i$, and
  \item[] For $1\leq i\leq n$, $S_B\rightarrow x_i a_i$.
\end{itemize}
Now, argue that $G$ is ambiguous if and only if the PCP instance 
$(A,B)$ has a solution
(thus, we may view the process of going from $(A,B)$ to 
$G$ as the desired mapping reduction). \hfill $\Box$
  \end{enumerate}

  %---
  \begin{enumerate}
  \item 
    Using the mapping reduction proposed in Exercise 15.2.3, 
    argue that
    if the PCP system has a solution, then the mapped object---the CFG---is
    ambiguous.
    %
    It asks you to argue more formally as follows in your answer:
    
    \begin{enumerate}
    \item 
      ``Suppose the given
      PCP system {\bf has} a solution;
     then a direct consequence of this
      is [THIS].'' Here, in your answer, you have to say what it means for
      a {\em common} string being parsed using the grammar resulting from
      this mapping reduction. Elaborate on ``[THIS]'' in 3-4 tight sentences.


    \item The translated grammar must then have [THIS] property with respect
      to ambiguity. Write that down formally.
      
    \end{enumerate}
    
  \item Show that if the PCP system has no solution,
    then the mapped object (CFG)
    is not ambiguous.
    %
    Argue the following carefully: suppose the PCP instance has no solution;
    could the generated CFG still admit some common string that produces
    two parse trees? What rules that out?



  \end{enumerate}

\item[] {\bf Solution}

  
\begin{itemize}
\item We design a mapping reduction that takes a PCP instance and emits the desired CFG. 

\item By our design, we can see that we have woven the PCP instance tile upper
  and lower halves into two alternate production schemes that split-off at $S_A$ and $S_B$. 

\item If the PCP instance has a solution, we can see a way to force ambiguity by forcing $S_A$ 
  and $S_B$ to produce the same output string,
  namely a sequence of $w_i$ followed by $a_i$ or  a sequence of $x_i$ followed by $a_i$. 

\item But since the $w_i$ and $x_i$ sequences are the same (PCP instance has a solution), we will obtain it only by matching the $a_i$ sequences; this shows that a parallel production scheme is to be taken via $S_A$ and $S_B$.

\item[] {\em In more detail,}

\item We can 
  see that if the PCP instance has no solution, there is no alternate production sequence that will match up the derived strings (the $a_i$ tails must still match). 

\item Thus, there will be no string that has two parses.

\item The $a_i$ symbols are not present in $\Sigma$.
 
\item Thus, any time a string has two parses, the "tail" $a_i$ sequences
  must agree.
  
\item Further, the "head" sequences in $w_i$ and $x_i$
  must agree, which corresponds to the PCP instance having a
  solution.


\item If the PCP instance has a solution, the mapped CFG will admit a common 
  $w_i$ and $x_i$ sequence. The tail $a_i$ sequence can be matched as well.
 
\item This results in an ambiguous parse for some string that is derived by concatenating
  the PCP instance solution with a corresponding $a_i$ tail sequence.

\item Suppose the PCP instance has no solution. 

\item Thus, there is no arrangement of the individual $w_i$s and matching $x_i$s
  such that the $w_i$ sequence agrees with the $x_i$ sequence. 

\item Is it still possible that there is a common string generated by the CFG?
  This is impossible as the $w_i$ and $x_i$ do not have common symbols over the $a_i$. 

\item Thus the only way common strings can be formed is ruled out by the $w_i$ and $x_i$ sequence matches being ruled out -- or that the CFG never generates ambiguous parses.
\end{itemize}



\item Review the P-time mapping reduction from 3SAT to Clique.

\item Show that if a 3CNF formula is a contradiction, then for any variable assignment,
  there is one clause that attains the value assignment ``TTT'' for its
  literals, and another clause that attains the value assignment ``FFF''

\item Review the P-time mapping reduction from 3SAT to Three-color.



\end{enumerate}

\end{document}


